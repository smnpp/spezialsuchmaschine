\documentclass[a4paper]{article}

%====================== PACKAGES ======================

\usepackage[french]{babel}
\usepackage[utf8x]{inputenc}
\usepackage{float}
\usepackage{amsmath}
\usepackage{graphicx}
\usepackage[colorinlistoftodos]{todonotes}
\usepackage{url}
\usepackage{hyperref}
\usepackage{array}
\usepackage{tabularx}
\usepackage{setspace}
\usepackage{abstract}
\usepackage[T1]{fontenc}
\usepackage[top=2cm, bottom=2cm, left=2cm, right=2cm]{geometry}
\usepackage{subfig}

%====================== INFORMATION ET REGLES ======================

%rajouter les numérotation pour les \paragraphe et \subparagraphe
\setcounter{secnumdepth}{4}
\setcounter{tocdepth}{4}

%======================== DEBUT DU DOCUMENT ========================

\begin{document}

%régler l'espacement entre les lignes
\newcommand{\HRule}{\rule{\linewidth}{0.5mm}}

%=========================== PAGE DE GARDE ===========================

\begin{center}
% Upper part of the page
\includegraphics[width=0.35\textwidth]{./logo}~\\[2cm]
\vspace{3cm}
% Title
\textsc{\LARGE INSA Lyon}\\[0.5cm]
\textsc{\Large Département Informatique}\\[0.5cm]
\HRule \\[0.4cm]
{\huge \bfseries Projet Web Sémantique\\
Spezialsuchmaschine\\[0.4cm] }
\HRule \\[1.5cm]
% Author and supervisor
\begin{minipage}{0.4\textwidth}
\begin{flushleft} \large
\emph{Auteurs:} \\
Hazim \textsc{Asri}\\
Nihal \textsc{Boutadghart}\\
Malte \textsc{Camier}\\
Jassir \textsc{Habba}\\
Junior \textsc{Noukam}\\
Simon \textsc{Perret}\\
\end{flushleft}
\end{minipage}
\begin{minipage}{0.4\textwidth}
\begin{flushright} \large
\emph{Professeurs:} \\
M. \textsc{Bento}\\
Mme. \textsc{Calabretto}\\
\end{flushright}
\end{minipage}
\vfill
% Bottom of the page
{\large \today}
    
\end{center}


%======================== TABLE DES MATIERES ========================
\newpage
~
\thispagestyle{empty}

\tableofcontents
\thispagestyle{empty}
\setcounter{page}{0}

\renewcommand{\arraystretch}{1.5}

%====================== INCLUSION DES PARTIES ======================

\newpage

% Introduction
\section{Introduction}
Le projet \textbf{Web Sémantique} réalisé dans le cadre de l'UE 4-IF-WS avait pour objectif principal de concevoir un moteur de recherche spécialisé exploitant les technologies du Web Sémantique. 

Dans ce projet, nous avons choisi de nous concentrer sur les \textit{marques automobiles allemandes} et leurs modèles, en utilisant la base de données DBpedia comme source d'information.

Les objectifs spécifiques étaient :
\begin{itemize}
    \item Explorer et interroger des données RDF via des requêtes SPARQL.
    \item Restituer les résultats de manière interactive et intuitive.
    \item Proposer une interface claire et accessible pour naviguer dans les résultats.
\end{itemize}

\newpage

% Technologies et outils utilisés
\section{Technologies et Outils Utilisés}
\subsection{Technologies Principales}
\begin{itemize}
    \item \textbf{SPARQL} : Pour effectuer des requêtes sémantiques sur DBpedia.
    \item \textbf{DBpedia} : Base de données RDF pour récupérer des informations liées aux marques automobiles.
    \item \textbf{HTML/CSS} : Structure et mise en forme de l'interface utilisateur.
    \item \textbf{JavaScript} : Pour la logique de l'application et les interactions dynamiques.
\end{itemize}

\subsection{Outils Utilisés}
\begin{itemize}
    \item \textbf{Visual Studio Code} : IDE principal pour le développement.
    \item \textbf{Git} : Gestion de version et collaboration.
    \item \textbf{Chrome DevTools} : Débogage et optimisation de l'interface utilisateur.
\end{itemize}

\newpage

% Architecture du projet
\section{Architecture du Projet}
\subsection{Organisation des Fichiers}
Le projet est structuré comme suit :
\begin{verbatim}
/-- index.html
/-- marque.html
/-- styles/
    |-- main.css
    |-- marque.css
/-- scripts/
    |-- requetes.js
    |-- marque.js
    |-- recherche.js
\end{verbatim}

\subsection{Flux de Données}
\begin{itemize}
    \item L'utilisateur saisit une requête dans la barre de recherche.
    \item Une requête SPARQL est générée dynamiquement en JavaScript.
    \item Les données sont récupérées depuis DBpedia et affichées sous forme de cartes ou de listes.
\end{itemize}

\newpage

% Fonctionnalités
\section{Fonctionnalités}
\subsection{Recherche Interactive}
\begin{itemize}
    \item Recherche dynamique des marques et modèles en fonction des lettres saisies.
    \item Résultats affichés dans un conteneur avec auto-complétion.
\end{itemize}

\subsection{Affichage des Marques}
\begin{itemize}
    \item Liste des marques allemandes triées par ordre alphabétique.
    \item Informations supplémentaires disponibles pour chaque marque (année de fondation, logo, etc.).
\end{itemize}

\subsection{Affichage des Modèles}
\begin{itemize}
    \item Informations détaillées sur les modèles : description, moteur, année de production, etc.
    \item Images illustrant les modèles.
\end{itemize}

\newpage

% Problèmes rencontrés
\section{Problèmes Rencontrés}
\begin{itemize}
    \item \textbf{Dépendance à DBpedia} : Temps de réponse variable et indisponibilité occasionnelle.
    \item \textbf{Complexité des Requêtes SPARQL} : Syntaxe délicate pour des résultats spécifiques.
    \item \textbf{Résultats incomplets} : Certaines données manquantes ou mal formatées.
\end{itemize}

\textbf{Solutions} :
\begin{itemize}
    \item Prévoir une vidéo de démonstration en cas de panne de DBpedia.
    \item Améliorer la gestion des erreurs et des données manquantes dans le code.
\end{itemize}

\newpage

% Réflexion sur le Web Sémantique
\section{Réflexion sur le Web Sémantique}
\subsection{Avantages}
\begin{itemize}
    \item Accès à des données interconnectées et structurées.
    \item Possibilité de requêtes complexes pour des besoins spécifiques.
\end{itemize}

\subsection{Inconvénients}
\begin{itemize}
    \item Complexité pour les débutants.
    \item Dépendance aux bases externes et à leur disponibilité.
    \item Difficultés liées au manque de standardisation dans certains domaines.
\end{itemize}

\newpage

% Conclusion
\section{Conclusion}
\begin{itemize}
    \item \textbf{Bilan} : Création réussie d'un moteur de recherche spécialisé pour les marques automobiles allemandes.
    \item \textbf{Compétences Acquises} :
    \begin{itemize}
        \item Maîtrise des requêtes SPARQL.
        \item Intégration des données RDF dans une application web.
        \item Développement d'une interface utilisateur interactive.
    \end{itemize}
    \item \textbf{Perspectives} :
    \begin{itemize}
        \item Ajouter des données supplémentaires et améliorer l'interface.
        \item Explorer d'autres bases de données sémantiques.
    \end{itemize}
\end{itemize}

\newpage


\end{document}